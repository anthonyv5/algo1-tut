\documentclass{scrartcl}

\usepackage[ngerman]{babel}
\usepackage{mathtools}
\usepackage{amssymb}
\usepackage{amsmath}
\usepackage{relsize}
\usepackage[normalem]{ulem}

\newcommand{\defeq}{\vcentcolon=}
\newcommand{\eqdef}{=\vcentcolon}
\newcommand{\bigO}{\ensuremath{\mathcal{O}}}

\begin{document}
\section{Bin"arsuche: Invarianten}

Sei $n = 2^k$ die L"ange des Eingabearrays $array$.

Seien $l_t$ bzw. $r_t$ die Werte der Variablen $l$ bzw. $r$ nach der $t$-ten Iteration (nur definiert, falls der Algorithmus auch
$\geq t$ Iterationen macht). $l_0$ und $r_0$ seien die Anfangswerte von $l$ und $r$.
Sei $i$ der eindeutig bestimmte Index mit $array[i] = x$.

Sinnvolle Invarianten k"onnten z.B. sein:

\begin{align}
r_t - l_t = 2^{k-t} \ \ \forall t \leq k \\
l_t \leq i < r_t \ \ \forall t \leq k
\end{align}

Daraus ergeben sich folgende Folgerungen:

\begin{itemize}
\item Aus (1) folgt, dass $r_k - l_k = 2^0 = 1$, also stoppt der Algorithmus nach sp"atestens $k$ Iterationen.
Er stoppt sogar nach genau $k$ Iterationen, da $r_t - l_t > 1 \ \ \forall t < k$
\item Falls der Algorithmus terminiert (was er nach (1) tut), gilt am Ende des Algorithmus:
      $r_k - l_k = 1$, also $l_k \leq i < r_k = l_k + 1$. Damit ist $l_k = i$ und in L10 wird
      das korrekte Ergebnis zur"uckgegeben
\end{itemize}

Der "Ubersichtlichkeit halber werden (1) und (2) hier getrennt bewiesen. Man kann auch
beide Invarianten auf einmal beweisen, was etwas Schreibarbeit spart. \\

\noindent\textbf{Beweis (1)}\\

Der Beweis erfolgt per vollst"andiger Induktion "uber $t$.\\

\noindent\textbf{IA}: Sei $t = 0$. Dann gilt aufgrund von L2-3: $r_t - l_t = 2^k - 0$ \\
\textbf{IS}: Sei $t \in \mathbb{N}, t \leq k - 1$ und $r_t - r_l = 2^{k-t}$. Zu zeigen: $r_{t+1} - l_{t+1} = 2^{k-(t+1)} = \frac{1}{2}\cdot 2^{k-t}$ \\

Sei zun"achst $m_{t+1} \defeq \frac{1}{2}(l_t + r_t)$ der Wert von $m$, der innerhalb der $(t+1)$ten Iteration berechnet wird.
Nach IV ist $m$ eine nat"urliche Zahl, denn $l_t$ und $r_t$ sind durch $2$ teilbar.

\uline{Fall 1}: $x < array[m]$

Dann setzt der Algorithmus also $l_{t+1} = l_t$ sowie $r_{t+1} = m_{t+1}$. Also gilt:
\[r_{t+1} - l_{t+1} = \frac{1}{2}(l_t + r_t) - l_t = \frac{1}{2}(r_t - l_t) \overset{IV}{=} \frac{1}{2}\cdot 2^{k-t}\]

\uline{Fall 2}: $x \geq array[m]$

I.d.F setzt der Algorithmus $l_{t+1} = m_{t+1}$ sowie $r_{t+1} = r_t$. Also gilt:
\[r_{t+1} - l_{t+1} = r_t - \frac{1}{2}(l_t + r_t) = \frac{1}{2}(r_t - l_t) \overset{IV}{=} \frac{1}{2}\cdot 2^{k-t}\] \\[3em]

\noindent\textbf{Beweis (2)}\\

Der Beweis erfolgt per vollst"andiger Induktion "uber $t$.\\

\noindent\textbf{IA}: Sei $t = 0$. Dann gilt: $l_t = 0 \leq i < n = r_t$ \\
\textbf{IS}: Sei $t \in \mathbb{N}, t \leq k - 1$ und $l_t \leq i < r_t$. Zu zeigen: $l_{t+1} \leq i < r_{t+1}$ \\

Sei zun"achst $m_{t+1} \defeq \frac{1}{2}(l_t + r_t)$ der Wert von $m$, der innerhalb der $(t+1)$ten Iteration berechnet wird.
Nach IV ist $m$ eine nat"urliche Zahl, denn $l_t$ und $r_t$ sind durch $2$ teilbar.

\uline{Fall 1}: $x < array[m]$

Da $array$ aufsteigend sortiert ist, gilt dann auch $i < m$.
Der Algorithmus setzt $l_{t+1} = l_t$ sowie $r_{t+1} = m_{t+1}$. Also gilt:
\[l_{t+1} = l_t \overset{IV}{\leq} i < m = r_{t+1}\]

\uline{Fall 2}: $x \geq array[m]$

Da $array$ aufsteigend sortiert ist, gilt dann auch $i \geq m$.
Der Algorithmus setzt $l_{t+1} = m_{t+1}$ sowie $r_{t+1} = r_t$. Also gilt:
\[l_{t+1} = m \leq i \overset{IV}{<} r_t = r_{t+1}\]

\end{document}
