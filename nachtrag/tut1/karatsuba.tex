\documentclass{scrartcl}

\usepackage[ngerman]{babel}
\usepackage{mathtools}

\newcommand{\defeq}{\vcentcolon=}
\newcommand{\eqdef}{=\vcentcolon}
\newcommand{\bigO}{\ensuremath{\mathcal{O}}}

\begin{document}
\section{Karatsuba}
Seien $a$, $b$ Zahlen mit $n = 2^k = 2m$ Bits mit

\begin{align*}
a &= \underbrace{a^{(2m-1)}a^{(2m-2)}\cdots a^{(m)}}_{\eqdef a_1}\underbrace{a^{(m-1)} \cdots a^{(0)}}_{\eqdef a_0} \\
b &= \underbrace{b^{(2m-1)}b^{(2m-2)}\cdots b^{(m)}}_{\eqdef b_1}\underbrace{b^{(m-1)} \cdots b^{(0)}}_{\eqdef b_0}
\end{align*}

Wobei $a^{(i)}$ und $b^{(j)}$ die Ziffern (Bits) der Zahl sind. Dann gilt:

\begin{align*}
a &= a_1\cdot 2^m + a_0 \\
b &= b_1\cdot 2^m + b_0 \\[1em]
\implies a \cdot b &= a_1 b_1\cdot 2^{2m} + a_1 b_0 2^m + a_0 b_1 2^m + a_0 b_0 \\
&= \underbrace{a_1 b_1}_{\eqdef c_{11}}\cdot 2^{2m} + \underbrace{a_0 b_0}_{\eqdef c_{00}}
                            + \underbrace{(a_1 b_0 + a_0 b_1)}_{\eqdef x}\cdot 2^m
\end{align*}

Um $a_1 b_0$ und $a_0 b_1$ zu berechnen, br"auchten wir bei einer naiven Implementierung 2 rekursive Aufrufe des
Multiplikationsalgorithmus. Wir nutzen aber nun aus, dass gilt:

\[(a_0 + a_1)(b_0 + b_1) = a_0 b_0 + a_0 b_1 + a_1 b_0 + a_1 a_1 = c_{00} + c_{11} + x \]

Also berechnen wir rekursiv:

\begin{itemize}
\item $c_{00}$
\item $c_{11}$
\item $y \defeq (a_0 + a_1)(b_0 + b_1)$
\end{itemize}

Und errechnen dann $a \cdot b$ ohne weitere Multiplikationen mittels der Formel

\[ a \cdot b = c_{00} + c_{11} + \left(y - c_{00} - c_{11}\right) \]

Laufzeit:

\[ T(n) = 3\cdot T\left(\frac{n}{2}\right) + \bigO(n) = \bigO(n^{log_2 3}) = \bigO(n^{1,59})\]

\end{document}
