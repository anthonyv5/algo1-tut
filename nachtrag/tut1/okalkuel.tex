\documentclass{scrartcl}

\usepackage[ngerman]{babel}
\usepackage{mathtools}
\usepackage{amssymb}
\usepackage{amsmath}
\usepackage{relsize}

\newcommand{\defeq}{\vcentcolon=}
\newcommand{\eqdef}{=\vcentcolon}
\newcommand{\bigO}{\ensuremath{\mathcal{O}}}

\newtheorem{lemma}{Lemma}

\begin{document}
\section{Beweise mit Landau-Symbolen}

\textbf{Behauptung}: $2^{2n} \not \in \bigO(2^n)$ \\
\textbf{Beweis}:   \\

\emph{Annahme}: $2^{2n} \in \bigO(2^n)$

Dann gilt nach Def.:

\begin{align}
\exists c > 0 \ \exists n_0: 2^{2n} \leq c\cdot 2^n \ \ \forall n \geq n_0
\end{align}

Wir formen um: $2^{2n} \leq c\cdot 2^n \iff 2^n \leq c \iff n \leq \log_2 c$. Dabei nutzen wir
die Monotonie des Logarithmus aus. Dies ist aber ein Widerspruch, denn f"ur $n > \min(\log_2 c, n_0)$ gilt
dann mit (1), dass $2^{2n} \leq c\cdot 2^n$ und gleichzeitig $2^{2n} > c\cdot 2^n$.
\\

Alternativ kann man auch mit analytischen Methoden zeigen, dass $2^{2n} \in \omega{2^n}$ gilt, denn

\[ \frac{2^{2n}}{2^n} = 2^n \ \xrightarrow{n \rightarrow \infty} \ \infty \]

\noindent\textbf{Behauptung}: $\sqrt{n} \in \omega(\log^2 n)$ \\
\textbf{Beweis}:   \\

Wir betrachten $\lim_{n \rightarrow \infty} \frac{\sqrt{n}}{\log^2 n}$. Da sowohl Z"ahler als auch
Nenner gegen $\infty$ gehen, d"urfen wir L'H\^{o}pitals Regel anwenden:

\begin{align*}
  &&\lim_{n \rightarrow \infty} \frac{\sqrt{n}}{\log^2 n} \\
  &\overset{\text{L'H\^{o}pital}}{=} &\lim_{n \rightarrow \infty} \frac{\frac{1}{2}n^{-\frac{1}{2}}}{2\log n \cdot \frac{1}{n}} \\
  &= &\frac{1}{4} \cdot \lim_{n \rightarrow \infty} \frac{\sqrt{n}}{\log n} \\
  &\overset{\text{L'H\^{o}pital}}{=} &\lim_{n \rightarrow \infty} \frac{\frac{1}{2}n^{-\frac{1}{2}}}{\frac{1}{n}} \\
  &= &\frac{1}{2}\lim_{n \rightarrow \infty} \sqrt{n} \\
  &\overset{\text{Stetigkeit Wurzelfktn}}{=} &\frac{1}{2}\sqrt{\lim_{n \rightarrow \infty} n} \\
  &=&\infty
\end{align*}

Die Gleichheitszeichen darf man nat"urlich, wenn man sehr korrekt ist, erst ganz am Ende einf"ugen, da
eine Pr"amisse von L'H\^{o}pital ist, dass der Grenzwert des transformierten Bruchs existiert oder
$= \infty$ ist.
\\

\noindent\textbf{Behauptung}: $n! \in \Omega(\sqrt{n^n})$ \\
\textbf{Beweis}:   \\

Beweis der Einfachkeit halber f"ur gerade $n$. Sei also $n = 2k$ mit $k \in \mathbb{N}$. Dann gilt:

\begin{align*}
  n! &= \underbrace{(2k)\cdot(2k-1)\cdots(k+1)}_{k \text{ Faktoren}}\cdot \underbrace{k \cdot (k-1) \cdots 3 \cdot 2}_{k - 1 \text{ Faktoren}} \cdot 1  \\
  &\geq (k+1)^k \cdot 2^{k-1} \\
  &= \frac{1}{2} \cdot (k+1)^k \cdot 2^k \\
  &\geq \frac{1}{2} \cdot \left(\frac{n}{2}\right)^{\frac{n}{2}} \cdot 2^{\frac{n}{2}} \\
  &= \frac{1}{2} n^{\frac{n}{2}} \\
  &= \frac{1}{2} {n^n}^{\frac{1}{2}} \\
  &= \frac{1}{2} \sqrt{n^n}
\end{align*}

\end{document}
