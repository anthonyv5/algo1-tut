\documentclass[18pt,t]{beamer}

\usepackage{sdq/templates/beamerthemekit}

\usepackage[utf8]{inputenc}
\usepackage[TS1,T1]{fontenc}
\usepackage{array}
\usepackage{xspace}
\usepackage{xcolor}
\usepackage{listings}
\usepackage{stmaryrd}
\usepackage{mathtools}
\usepackage{amssymb}
\usepackage{amsmath}
\usepackage[normalem]{ulem}
\usepackage[scaled]{beramono}

\input{../config}

\usepackage[citestyle=authoryear,bibstyle=numeric,hyperref,backend=biber]{biblatex}
\addbibresource{../references.bib}
\bibhang1em

\author{\authorName}
\institute[]{Institut für theoretische Informatik, Prof. Sanders}

\selectlanguage{ngerman}

\titleimage{title}

\newcommand{\bigO}{\ensuremath{\mathcal{O}}}
\newcommand{\defeq}{\vcentcolon=}
\newcommand{\eqdef}{=\vcentcolon}

\DeclarePairedDelimiter{\ceil}{\lceil}{\rceil}
\DeclarePairedDelimiter{\floor}{\lfloor}{\rfloor}

\beamertemplatenavigationsymbolsempty

% vertical table padding
\renewcommand{\arraystretch}{1.5}

\definecolor{colkeywords}{rgb}{0,0,0.4}
\definecolor{colcomment}{rgb}{0,0.5,0}
\definecolor{colstring}{rgb}{0.5,0,0}
\definecolor{colconst}{rgb}{0.9,0.1,0.8}
\definecolor{collinenums}{rgb}{0.5,0.5,0.5}
\definecolor{coldigit}{rgb}{0.9,0.1,0.8}

\lstset{
  basicstyle=\normalsize\ttfamily,
  numbers=left,
  numberstyle=\scriptsize,
  numbersep=-5pt,
  tabsize=4,
  framexleftmargin=1mm,
  xleftmargin=10mm,
  escapeinside={\%*}{*)},
  commentstyle=\color{colcomment},
  keywordstyle=\color{colkeywords}\bfseries,
  stringstyle=\color{colstring},
  numberstyle=\color{collinenums},
  literate=%
    {0}{{{\color{coldigit}0}}}1
    {1}{{{\color{coldigit}1}}}1
    {2}{{{\color{coldigit}2}}}1
    {3}{{{\color{coldigit}3}}}1
    {4}{{{\color{coldigit}4}}}1
    {5}{{{\color{coldigit}5}}}1
    {6}{{{\color{coldigit}6}}}1
    {7}{{{\color{coldigit}7}}}1
    {8}{{{\color{coldigit}8}}}1
    {9}{{{\color{coldigit}9}}}1
    {Ö}{{\"O}}1
    {Ä}{{\"A}}1
    {Ü}{{\"U}}1
    {ß}{{\ss}}2
    {ü}{{\"u}}1
    {ä}{{\"a}}1
    {ö}{{\"o}}1
}

\lstdefinelanguage{Pseudo}
{keywords={Function,return,assert,while,for,to,if,else,not},%
emph={FALSE,TRUE},
emphstyle=\color{colconst},
sensitive=true,%
comment=[l]{//},%
string=[b]",%
}


\lstset{language=Pseudo}


\title[Tutorium Algorithmen I]{Tutorium \tutNo, Algorithmen I}
\subtitle{Woche 4 -- Hashing, Sortieren}

\begin{document}

\begin{frame}
  \titlepage
\end{frame}

\section{Hashing}

\subsection{Hashing}
\begin{frame}
  \frametitle{"Ubersicht Hashing}
  \begin{itemize}
  \item Sei $m \in \mathbb{N}$ und $K$ ein beliebige Menge.
        Eine \textbf{Hashfunktion} ist eine Funktion $h : K \rightarrow \{0,\ldots,m-1\}$
  \item Eine \textbf{Hashkollision} bezeichnet ein Paar $x, y \in K$ mit $x \neq y$ und
        $h(x) = h(y)$
  \item W"unschenswert: komplett zuf"allige Hashfunktion \\
          $\rightarrow$ unrealistisch, da ineffizient
  \visible<2->{
  \item Eine \emph{Familie} $\mathscr{H}$ von Hashfunktionen hei"st
        \textbf{universell} $\iff$ f"ur ein zuf"alliges $h \in \mathscr{H}$ gilt:
        $\forall x, y \in K, x \neq y: \mathbb{P}(h(x) = h(y)) = \frac{1}{m}$
  }
  \end{itemize}
\end{frame}

\begin{frame}
  \frametitle{Hashtabellen}
  \begin{itemize}
  \item Verwalten eine Menge $M$ mit \textbf{Schl"usseln} aus $K$
  \item Wichtige Operationen
    \begin{itemize}
    \item \lstinline|find(key: K): M|
    \item \lstinline|insert(key: K, value: M)|
    \item evtl. \lstinline|remove(key: K)|
    \end{itemize}
  \item Oft zwei verschiedene Interfaces:
    \begin{itemize}
    \item Zur Verwaltung von Schl"usselmengen
             (z.B. \lstinline|HashSet| in Java)
    \item Zur Zuordnung Schl"ussel $\rightarrow$ Elemente
             (z.B. \lstinline|HashMap| in Java)
    \end{itemize}
  \end{itemize}
\end{frame}

\begin{frame}
  \frametitle{Hashtabellen (2)}
  \begin{itemize}
  \item Implementierung: Array der Gr"o"se $m$, $h$ bildet $K$ auf Slots der
        Tabelle ab. Kollisionen m"ussen behandelt werden
    \begin{itemize}
      \visible<2->{
      \item \textbf{Verlinkte Listen}: Jeder Slot verwaltet die Kollisionen in einer Liste
      }
      \visible<3->{
      \item \textbf{Linear probing}: pro Slot ein Element. Bei Kollision werden Slots
            genutzt
      }
    \end{itemize}
  \visible<4->{
  \item Bei geeigneter Hashfunktion und falls $m \in \Omega(n)$:
        $\bigO(1)$ erwartete Laufzeit f"ur Operationen
  }
  \end{itemize}
\end{frame}

\begin{frame}
  \frametitle{Aufgabe 1 (Hashing)}
  \begin{itemize}
  \item \textbf{Gegeben}: Zahl $x$ und Array $A = \langle A[0], A[1], \ldots, A[n-1] \rangle$
        von Ganzzahlen
  \item Finde Indices $i, j$ mit $A[i] + A[j] = x$ oder gib $\bot$ aus, falls
        diese nicht existieren
  \end{itemize}
\end{frame}

\begin{frame}[fragile]
  \frametitle{Aufgabe 1 (L"osung)}
  \begin{lstlisting}
  Function 2SUM(A, x)
      numberToIndex = new HashMap(A.length)
      for i := 0 to A.length - 1
          numberToIndex.insert(A[i], i)
      for i := 0 to A.length - 1
          j := numberToIndex.find(x - A[i])
          if j != %*$\bot$*)
              return (i, j)
      return %*$\bot$*)
  \end{lstlisting}
\end{frame}

\begin{frame}
  \frametitle{Aufgabe 2 (Hashing)}
  \begin{itemize}
  \item Ben"otigt wird eine Queue, die folgende Operationen unterst"utzt:
    \begin{itemize}
    \item \lstinline|enqueue(x: M)| und \lstinline|dequeue(): M| in $\bigO(1)$
    \item \lstinline|remove(x: M)| in \emph{erwartet} $\bigO(1)$
    \end{itemize}
  \item \emph{Hinweis}: Die Elemente $x$ aus $M$ k"onnen als Schl"ussel einer
        Hashtabelle verwendet werden
  \end{itemize}
\end{frame}

\begin{frame}[fragile]
  \frametitle{Aufgabe 2 (Hashing)}
  \begin{lstlisting}
  q : LinkedList
  map : HashMap<M, ListItem>

  Function enqueue(x: M)
      q.pushFront(x)
      map.insert(x, q.head)
  Function remove(x: M)
      item := map.find(x)
      item.remove()
      map.remove(x)
  Function dequeue() : M
      x := q.tail
      remove(x)
      return x
  \end{lstlisting}
\end{frame}

\section{Sortieren}
\subsection{"Ubersicht}
\begin{frame}
  \frametitle{Sortieren}
  \begin{itemize}
  \item \textbf{Problem}: Finde ,,sortierte'' Permutation einer Eingabefolge
                         $A = \langle A[0], A[1], \ldots, A[n-1] \rangle$
  \item Man darf: Elemente vergleichen ($A[i] \leq A[j], A[i] = A[j]$)
  \item Begriffe:
    \begin{itemize}
    \item \textbf{in-place}: Nur $\bigO(1)$ Zusatzspeicher wird ben"otigt
    \item \textbf{stabil}: Der Sortieralgorithmus erh"alt die Reihenfolge von
                           ,,gleichen'' Elementen
    \end{itemize}
  \end{itemize}
\end{frame}

\subsection{Algorithmen}

\begin{frame}
  \frametitle{Insertion sort}
  \begin{itemize}
  \item Intuitiver Ansatz (vgl. Kartenspiele): Sortieren durch Einf"ugen
        (\emph{Insertion sort})
  \begin{itemize}
  \item Beginne mit leerer Ergebnisliste $B = \langle\rangle$
  \item F"uge der Reihe nach die Elemente $A[0],\ldots,A[n-1]$ derart in $B$ ein,
        dass $B$ immer sortiert ist
  \end{itemize}
  \item Sehr effizient f"ur kleine ($n \leq 10$) und ,,fast'' sortierte Eingaben
  \end{itemize}
\end{frame}

\begin{frame}
  \frametitle{Merge sort}
  \begin{itemize}
  \item Divide and Conquer:
    \begin{itemize}
    \item Halbiere Liste
    \item Sortiere H"alften rekursiv (trivialer Basisfall: $n \leq 1$)
    \item F"uge sortierte Listen in $\bigO(n)$ zusammen (wie auf "UB2)
    \end{itemize}
  \item Mit Mastertheorem: $\Theta(n \cdot \log n)$
  \item Stabil, einfach zu implementieren, aber $\Omega(n)$ Zusatzspeicher
  \end{itemize}
\end{frame}

\begin{frame}
  \frametitle{Quicksort}
  \begin{itemize}
  \item Divide and Conquer mit Zufall:
    \begin{itemize}
    \item Teile Eingabe grob in zwei Teile anhand von Pivot p \\
          (linker Teil $< p$, rechter Teil $>p$)
    \item Sortiere rekursiv
    \end{itemize}
  \item nicht stabil
  \item Zusatzspeicher? \visible<2->{$\Omega(\log n)$}
  \item Best case + Average case (zuf"allige Permutation):
          $\bigO(n \cdot \log n)$
  \item Worst case $\Omega(n^2)$, abh"angig von Pivotwahl
    \begin{itemize}
    \item Bei Wahl des ersten/letzten Elements als Pivot?
    \item Bei Wahl des mittleren Elements als Pivot?
    \end{itemize}
  \end{itemize}
\end{frame}

\section{Selection}
\subsection{Selection}
\begin{frame}
  \frametitle{Das Auswahlproblem}
  \begin{itemize}
  \item Verwandt zu Sortierproblem: Gegeben $A = \langle A[0], A[1], \ldots, A[n-1] \rangle$,
        finde das \emph{k-kleinste} Element in $A$
  \item Mit Sortieren l"osbar, aber in Wirklichkeit ein einfacheres Problem!
  \item \emph{Spezialfall}: Finde das $\lfloor n/2 \rfloor$-te Element in $A$, den
        \textbf{Median}
  \item Kreativaufgabe:
    \begin{itemize}
    \item Wir haben einen Algorithmus \lstinline|MEDIAN| zur Verf"ugung,
          der uns den Median eines Arrays oder Teilarrays $A$ in
          $\bigO(n)$ berechnet \\[1em]
          (a) Bestimme das $\lfloor n/4 \rfloor$-kleinste
          Element in $A$ in $\bigO(n)$ (25\%-Perzentil)\\[0.3em]
          (b) Bestimme das $\lfloor n/3 \rfloor$-kleinste
          Element in $A$ in $\bigO(n)$ (33\%-Perzentil)
    \end{itemize}
  \end{itemize}
\end{frame}

\end{document}
