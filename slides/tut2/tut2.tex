\documentclass[18pt,t]{beamer}

\usepackage{sdq/templates/beamerthemekit}

\usepackage[utf8]{inputenc}
\usepackage[TS1,T1]{fontenc}
\usepackage{array}
\usepackage{xspace}
\usepackage{xcolor}
\usepackage{listings}
\usepackage{stmaryrd}
\usepackage{mathtools}
\usepackage{amssymb}
\usepackage{amsmath}
\usepackage[normalem]{ulem}
\usepackage[scaled]{beramono}

\input{../config}

\usepackage[citestyle=authoryear,bibstyle=numeric,hyperref,backend=biber]{biblatex}
\addbibresource{../references.bib}
\bibhang1em

\author{\authorName}
\institute[]{Institut für theoretische Informatik, Prof. Sanders}

\selectlanguage{ngerman}

\titleimage{title}

\newcommand{\bigO}{\ensuremath{\mathcal{O}}}
\newcommand{\defeq}{\vcentcolon=}
\newcommand{\eqdef}{=\vcentcolon}

\DeclarePairedDelimiter{\ceil}{\lceil}{\rceil}
\DeclarePairedDelimiter{\floor}{\lfloor}{\rfloor}

\beamertemplatenavigationsymbolsempty

% vertical table padding
\renewcommand{\arraystretch}{1.5}

\definecolor{colkeywords}{rgb}{0,0,0.4}
\definecolor{colcomment}{rgb}{0,0.5,0}
\definecolor{colstring}{rgb}{0.5,0,0}
\definecolor{colconst}{rgb}{0.9,0.1,0.8}
\definecolor{collinenums}{rgb}{0.5,0.5,0.5}
\definecolor{coldigit}{rgb}{0.9,0.1,0.8}

\lstset{
  basicstyle=\normalsize\ttfamily,
  numbers=left,
  numberstyle=\scriptsize,
  numbersep=-5pt,
  tabsize=4,
  framexleftmargin=1mm,
  xleftmargin=10mm,
  escapeinside={\%*}{*)},
  commentstyle=\color{colcomment},
  keywordstyle=\color{colkeywords}\bfseries,
  stringstyle=\color{colstring},
  numberstyle=\color{collinenums},
  literate=%
    {0}{{{\color{coldigit}0}}}1
    {1}{{{\color{coldigit}1}}}1
    {2}{{{\color{coldigit}2}}}1
    {3}{{{\color{coldigit}3}}}1
    {4}{{{\color{coldigit}4}}}1
    {5}{{{\color{coldigit}5}}}1
    {6}{{{\color{coldigit}6}}}1
    {7}{{{\color{coldigit}7}}}1
    {8}{{{\color{coldigit}8}}}1
    {9}{{{\color{coldigit}9}}}1
    {Ö}{{\"O}}1
    {Ä}{{\"A}}1
    {Ü}{{\"U}}1
    {ß}{{\ss}}2
    {ü}{{\"u}}1
    {ä}{{\"a}}1
    {ö}{{\"o}}1
}

\lstdefinelanguage{Pseudo}
{keywords={Function,return,assert,while,for,to,if,else,not},%
emph={FALSE,TRUE},
emphstyle=\color{colconst},
sensitive=true,%
comment=[l]{//},%
string=[b]",%
}


\lstset{language=Pseudo}


\title[Tutorium Algorithmen I]{Tutorium \tutNo, Algorithmen I}
\subtitle{Woche 2 -- Beweise, }

\begin{document}

\begin{frame}
  \titlepage
\end{frame}

\begin{frame}
  \frametitle{"Uberblick}
  \begin{itemize}
  \item Reflektion "UB1
  \item Einfache Datenstrukturen (Stacks, Queues)
    \begin{itemize}
    \item Amortisierte Analyse
    \item Deamortisierung
    \end{itemize}
  \item Hashing
  \item (Rekurrenzen)
  \end{itemize}
\end{frame}

\section{Reflektion "UB1}
\subsection{O-Kalk"ul}
\begin{frame}
  \frametitle{O-Kalk"ul}
  \begin{itemize}
  \item $f(n) = O(n)$ ist g"angig, $f(n) \neq O(n)$ nicht. Stattdessen: $f(n) \not \in O(n)$
  \item F"ur beliebige asymptotisch pos. Funktionen $f, g$ muss \uline{nicht} gelten,
        dass $f \in O(g)$ oder $g \in O(f)$. \\
        Gegenbeispiel: $f(n) \defeq |sin(n)|, g(n) \defeq |cos(n)|$
  \item $O(f(n)) + O(g(n))$ bedeutet \uline{nicht} Mengenvereinigung. Vielmehr gilt:
        $O(f(n)) + O(g(n)) = \{ a(n) + g(n) \ |\  a(n) \in O(f(n)), b(n) \in O(g(n)) \}$
  \item $f(n) \in O(g(n))$ l"asst sich i.A. \uline{nicht} "uber
        $\lim_{n \rightarrow \infty} |\frac{f(n)}{g(n)}|$
        ausdr"ucken, der Limes muss nicht einmal existieren.
        Stattdessen kann man $\limsup_{n \rightarrow \infty} |\frac{f(n)}{g(n)}|$ benutzen.
  \end{itemize}
\end{frame}

\subsection{Beweisf"uhrung}
\begin{frame}
  \frametitle{Beweisf"uhrung}
  \begin{itemize}
  \item ,,$A \implies B$'': aus Aussage $A$ folgt Aussage $B$
    \begin{itemize}
    \item Problematisch: ,,Sei $x = 2$. $\implies x > 1$''
    \item Besser: ,,Sei $x = 2$. Dann: $x > 1$'' \\[1em]
    \item Problematisch: ,,Annahme: $A$ gilt nicht
           $\implies B \implies C \ \lightning \implies A$''
    \item Besser: ,,Annahme: $A$ gilt nicht. Dann gilt: $B \implies C \ \lightning$.
          [Also ist die Annahme falsch und $A$ muss gelten.]''
    \end{itemize}
  \item Beweisrichtung muss richtig sein
    \begin{itemize}
    \item Problematisch: ,,Annahme: $1 > 2$. Dann gilt auch:
          $3 = 2 + 1 > 2$. $3 > 2$ ist wahr,
          \emph{also muss auch die Annahme wahr sein.}''
    \item Aus einer falschen Aussage folgt Beliebiges!
    \end{itemize}
  \end{itemize}
\end{frame}

\renewcommand{\arraystretch}{1.0}
\begin{frame}
  \frametitle{Beweistipps}
  \begin{itemize}
  \item Behauptung und Beweis explizit trennen
  \item ,,$\iff$'' eher selten verwenden, stattdessen ,,$\implies$''
  \item Schema \textbf{direkter Beweis}: \\
        Bekannt: $X \implies \ldots \implies$ Behauptung. $\square$
  \item Schema \textbf{indirekter Beweis}: \\
        Annahme: Behauptung falsch.
        Dann: $X \implies \ldots \implies Y \ \lightning \ \square$
  \end{itemize}

  \begin{center}
  \begin{tabular}{| l || l |} \hline
  \textbf{Behauptung} & \textbf{Beweis} \\ \hline
    $\forall x \in U: A(x)$
      & Sei $x \in U$ [beliebig, aber fest]. Zeige: $A(x)$.
      \\ \hline
    $\exists x \in U: A(x)$
      & Gib ein bestimmtes $x \in U$ an und zeige $A(x)$.
      \\ \hline
    $U \subseteq V$
      & Sei $x \in U$. Zeige, dass $x \in V$.
      \\ \hline
    $U = V$
      & Zeige $U \subseteq V$ und $V \subseteq U$.
      \\ \hline
    $\forall n \in \mathbb{N}, n \geq n_0: A(n)$
      & Zeige $A(n_0)$ und $\forall n \geq n_0: A(n) \implies A(n+1)$
      \\ \hline
  \end{tabular}
  \end{center}
\end{frame}
\renewcommand{\arraystretch}{1.5}

\section{Datenstrukturen}
\subsection{Einfache Datenstrukturen}
\begin{frame}
  \frametitle{Einfache Datenstrukturen}
  \begin{definition}
    Eine \textbf{Datenstruktur} ist eine Repr"asentation von Daten in einem
    Rechnermodell, zusammen mit Algorithmen zur Modifikation und Abfrage der
    verwalteten Daten (\emph{Operationen})
  \end{definition}
  \begin{itemize}
  \item Elementare Bausteine:
    \begin{itemize}
      \item Arrays
      \item Zeiger
      \item Zusammengesetzte Objekte (Records/Structs/Nodes)
    \end{itemize}
  \item Werden zusammengesetzt zu komplexeren Datenstrukturen (verlinkte Listen,
        Stacks, Queues, Suchb"aume...)
  \end{itemize}
\end{frame}

\begin{frame}
  \frametitle{Einfache Datenstrukturen (2)}
  \begin{itemize}
    \item \textbf{Unbeschr"ankte/Dynamische Arrays}: Erlauben wahlfreien Zugriff in $\bigO(1)$
          und \lstinline{pushBack}/\lstinline{popBack} in amortisiert $\bigO(1)$
    \item \textbf{Doppelt verkettete Listen}: Erlauben
          \lstinline|pushFront|/\lstinline|popFront|/
          \lstinline|pushBack|/\lstinline|popBack|/\lstinline|remove| in $\bigO(1)$
    \item \textbf{Einfach verlinkte Listen}: Erlauben
          \lstinline|pushFront|/\lstinline|popFront|/\lstinline|removeAfter|
          in $\bigO(1)$
  \end{itemize}
\end{frame}

\subsection{Zusammensetzen von Datenstrukturen}
\begin{frame}
  \frametitle{Zusammengesetzte Datenstrukturen}
  \begin{itemize}
  \item \textbf{Gegeben}: Datenstruktur ,,Stack'' mit Operationen \lstinline|push|,
        \lstinline|pop| und \lstinline|isEmpty|
  \item Realisiere die Datenstruktur ,,Queue'' mit den Operationen \lstinline|enqueue|
        und \lstinline|dequeue|
  \end{itemize}
\end{frame}

\begin{frame}[fragile]
  \frametitle{Zusammengesetzte Datenstrukturen (2)}
  \begin{itemize}
  \item \textbf{L"osung}: Benutze 2 Stacks \lstinline|S1| und \lstinline|S2|.
        \lstinline|enqueue| pusht auf \lstinline|S1|, \lstinline|dequeue| poppt von
        \lstinline|S2|. Falls \lstinline|S2| leer wird, wird er wieder mit dem Inhalt
        von \lstinline|S1| gef"ullt

  \begin{lstlisting}
  Function enqueue(x)
      S1.push(x)

  Function dequeue()
      if S2.isEmpty()
          while not S1.empty()
              S2.push(S1.pop())
      return S2.pop()
  \end{lstlisting}

  \item \textbf{Behauptung}: \lstinline|enqueue|/\lstinline|dequeue| haben amortisierte
                             Laufzeit $\bigO(1)$
  \end{itemize}
\end{frame}

\subsection{Demortisierung}
\begin{frame}
  \frametitle{Deamortisierung}
  \begin{itemize}
  \item Wir wollen Worst-Case-Laufzeit f"ur jede Einzeloperation garantieren
    \begin{itemize}
    \item Beispiel: Anforderungen an die maximale Verarbeitungszeit einer Anfrage
    \end{itemize}
  \item \textbf{Beispiel}: Gesucht ist eine Datenstruktur,
        die deterministisch die Operationen eines dynamischen Arrays unterst"utzt:
    \begin{itemize}
    \item \lstinline|pushBack| in $\bigO(1)$
    \item wahlfreier Zugriff in $\bigO(\log n)$
    \end{itemize}
  \item Ebenfalls m"oglich:
    \begin{itemize}
    \item \lstinline|pushBack| in $\bigO(\log n)$
    \item wahlfreier Zugriff in $\bigO(1)$
    \end{itemize}
  \item Allokationen sollen dabei in $\bigO(1)$ m"oglich sein
  \end{itemize}
\end{frame}

\begin{frame}
  \frametitle{Deamortisierung (2)}
  \begin{itemize}
  \item L"osung:
    \begin{itemize}
    \item Doppelt verlinkte Liste von Arrays der L"ange $2^0$, $2^1$, $2^2$, $2^3$, $\ldots$
    \item Zeiger auf letztes Array
    \item Bei "Uberlauf: neues Array allokieren und an Liste anf"ugen
    \end{itemize}
  \item F"ur zweite Variante:
    \begin{itemize}
    \item Array von Arrays der L"ange $2^0$, $2^1$, $2^2$, $2^3$, $\ldots$
    \item Das Element mit Index $i \in \{1,\ldots,n\}$ befindet sich in Array Nummer
          $\floor{\log_2 i}$
          (kann in $\bigO(1)$ berechnet werden)
    \item Bei "Uberlauf muss Basisarray vergr"o"sert werden (es hat L"ange $\bigO(\log n)$)
    \end{itemize}
  \end{itemize}
\end{frame}

\section{Hashing}

\subsection{Hashing}
\begin{frame}
  \frametitle{Hashing}
  \begin{itemize}
  \item Sei $m \in \mathbb{N}$ und $U$ ein beliebige Menge.
        Eine Hashfunktion ist eine Funktion $h : U \rightarrow \{0,\ldots,m-1\}$
  \item W"unschenswert w"are eine komplett zuf"allige Hashfunktion $h$
  \item Eine Familie $\mathscr{H}$ von Hashfunktionen hei"st \textbf{universell} $\iff$
        f"ur ein zuf"alliges $h \in \mathscr{H}$ gilt:
        $\forall x, y \in U, x \neq y: \mathbb{P}(h(x) = h(y)) = \frac{1}{m}$
  \item Hashtabelle: Array der Gr"o"se $m$, $h$ bildet $U$ auf Slots der
        Tabelle ab. Kollisionen m"ussen behandelt werden
    \begin{itemize}
      \item \textbf{Offene Adressierung}: pro Slot ein Element. Bei Kollision wird zum
            Beispiel n"achster Slot versucht (Linear Probing)
      \item \textbf{Verlinkte Listen}: Jeder Slot verwaltet die Kollisionen in einer Liste
    \end{itemize}
  \item Bei geeigneter Hashfunktion und falls $m \in \Omega(n)$:
        $\bigO(1)$ erwartete Laufzeit f"ur Operationen
        \lstinline|find|/\lstinline|insert|/\lstinline|remove|
  \end{itemize}
\end{frame}

\begin{frame}
  \frametitle{Hashing (2)}
  \begin{itemize}
  \item \textbf{Gegeben}: Zahl $x$ und Array $A = \langle A[0], A[1], \ldots, A[n-1] \rangle$
        von Ganzzahlen, mit $A[i-1] \leq A[i] \ \ \forall i \in \{1,\ldots,n-1\}$ ($A$ aufsteigend sortiert)
  \item Finde Elemente $A[i], A[j]$ mit $A[i] + A[j] = x$ oder gib $\bot$ aus, falls
        diese nicht existieren \\
    \ \ (a) in \emph{erwarteter} Laufzeit $\bigO(n)$ \\
    \ \ (b) in \emph{erwarteter} Laufzeit $\bigO(n \cdot \log n)$ \\
    \ \ (c) in \emph{Worst-Case}-Laufzeit $\bigO(n)$
  \end{itemize}
\end{frame}
\end{document}
