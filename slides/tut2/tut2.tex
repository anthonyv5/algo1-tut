\documentclass[18pt,t]{beamer}

\usepackage{sdq/templates/beamerthemekit}

\usepackage[utf8]{inputenc}
\usepackage[TS1,T1]{fontenc}
\usepackage{array}
\usepackage{xspace}
\usepackage{xcolor}
\usepackage{listings}
\usepackage{stmaryrd}
\usepackage{mathtools}
\usepackage{amssymb}
\usepackage{amsmath}
\usepackage[normalem]{ulem}
\usepackage[scaled]{beramono}

\input{../config}

\usepackage[citestyle=authoryear,bibstyle=numeric,hyperref,backend=biber]{biblatex}
\addbibresource{../references.bib}
\bibhang1em

\author{\authorName}
\institute[]{Institut für theoretische Informatik, Prof. Sanders}

\selectlanguage{ngerman}

\titleimage{title}

\newcommand{\bigO}{\ensuremath{\mathcal{O}}}
\newcommand{\defeq}{\vcentcolon=}
\newcommand{\eqdef}{=\vcentcolon}

\DeclarePairedDelimiter{\ceil}{\lceil}{\rceil}
\DeclarePairedDelimiter{\floor}{\lfloor}{\rfloor}

\beamertemplatenavigationsymbolsempty

% vertical table padding
\renewcommand{\arraystretch}{1.5}

\definecolor{colkeywords}{rgb}{0,0,0.4}
\definecolor{colcomment}{rgb}{0,0.5,0}
\definecolor{colstring}{rgb}{0.5,0,0}
\definecolor{colconst}{rgb}{0.9,0.1,0.8}
\definecolor{collinenums}{rgb}{0.5,0.5,0.5}
\definecolor{coldigit}{rgb}{0.9,0.1,0.8}

\lstset{
  basicstyle=\normalsize\ttfamily,
  numbers=left,
  numberstyle=\scriptsize,
  numbersep=-5pt,
  tabsize=4,
  framexleftmargin=1mm,
  xleftmargin=10mm,
  escapeinside={\%*}{*)},
  commentstyle=\color{colcomment},
  keywordstyle=\color{colkeywords}\bfseries,
  stringstyle=\color{colstring},
  numberstyle=\color{collinenums},
  literate=%
    {0}{{{\color{coldigit}0}}}1
    {1}{{{\color{coldigit}1}}}1
    {2}{{{\color{coldigit}2}}}1
    {3}{{{\color{coldigit}3}}}1
    {4}{{{\color{coldigit}4}}}1
    {5}{{{\color{coldigit}5}}}1
    {6}{{{\color{coldigit}6}}}1
    {7}{{{\color{coldigit}7}}}1
    {8}{{{\color{coldigit}8}}}1
    {9}{{{\color{coldigit}9}}}1
    {Ö}{{\"O}}1
    {Ä}{{\"A}}1
    {Ü}{{\"U}}1
    {ß}{{\ss}}2
    {ü}{{\"u}}1
    {ä}{{\"a}}1
    {ö}{{\"o}}1
}

\lstdefinelanguage{Pseudo}
{keywords={Function,return,assert,while,for,to,if,else,not},%
emph={FALSE,TRUE},
emphstyle=\color{colconst},
sensitive=true,%
comment=[l]{//},%
string=[b]",%
}


\lstset{language=Pseudo}


\title[Tutorium Algorithmen I]{Tutorium \tutNo, Algorithmen I}
\subtitle{Woche 2 -- Beweise, }

\begin{document}

\begin{frame}
  \titlepage
\end{frame}

\section{Reflektion "UB1}
\subsection{O-Kalk"ul}
\begin{frame}
  \frametitle{O-Kalk"ul}
  \begin{itemize}
  \item $f(n) = O(n)$ ist g"angig, $f(n) \neq O(n)$ nicht. Stattdessen: $f(n) \not \in O(n)$
  \item F"ur beliebige asymptotisch pos. Funktionen $f, g$ muss \uline{nicht} gelten,
        dass $f \in O(g)$ oder $g \in O(f)$. \\
        Gegenbeispiel: $f(n) \defeq |sin(n)|, g(n) \defeq |cos(n)|$
  \item $O(f(n)) + O(g(n))$ bedeutet \uline{nicht} Mengenvereinigung. Vielmehr gilt:
        $O(f(n)) + O(g(n)) = \{ a(n) + g(n) \ |\  a(n) \in O(f(n)), b(n) \in O(g(n)) \}$
  \item $f(n) \in O(g(n))$ l"asst sich i.A. \uline{nicht} "uber
        $\lim_{n \rightarrow \infty} |\frac{f(n)}{g(n)}|$
        ausdr"ucken, der Limes muss nicht einmal existieren.
        Stattdessen kann man $\limsup_{n \rightarrow \infty} |\frac{f(n)}{g(n)}|$ benutzen.
  \end{itemize}
\end{frame}

\subsection{Beweisf"uhrung}
\begin{frame}
  \frametitle{Beweisf"uhrung}
  \begin{itemize}
  \item ,,$A \implies B$'': aus Aussage $A$ folgt Aussage $B$
    \begin{itemize}
    \item Problematisch: ,,Sei $x = 2$. $\implies x > 1$''
    \item Besser: ,,Sei $x = 2$. Dann: $x > 1$'' \\[1em]
    \item Problematisch: ,,Annahme: $A$ gilt nicht
           $\implies B \implies C \ \lightning \implies A$''
    \item Besser: ,,Annahme: $A$ gilt nicht. Dann gilt: $B \implies C \ \lightning$.
          [Also ist die Annahme falsch und $A$ muss gelten.]''
    \end{itemize}
  \item Beweisrichtung muss richtig sein
    \begin{itemize}
    \item Problematisch: ,,Annahme: $1 > 2$. Dann gilt auch:
          $3 = 2 + 1 > 2$. $3 > 2$ ist wahr,
          \emph{also muss auch die Annahme wahr sein.}''
    \item Aus einer falschen Aussage folgt Beliebiges!
    \end{itemize}
  \end{itemize}
\end{frame}

\renewcommand{\arraystretch}{1.0}
\begin{frame}
  \frametitle{Beweistipps}
  \begin{itemize}
  \item Behauptung und Beweis explizit trennen
  \item ,,$\iff$'' eher selten verwenden, stattdessen ,,$\implies$''
  \item Schema \textbf{direkter Beweis}: \\
        Bekannt: $X \implies \ldots \implies$ Behauptung. $\square$
  \item Schema \textbf{indirekter Beweis}: \\
        Annahme: Behauptung falsch.
        Dann: $X \implies \ldots \implies Y \ \lightning \ \square$
  \end{itemize}

  \begin{center}
  \begin{tabular}{| l || l |} \hline
  \textbf{Behauptung} & \textbf{Beweis} \\ \hline
    $\forall x \in U: A(x)$
      & Sei $x \in U$ [beliebig, aber fest]. Zeige: $A(x)$.
      \\ \hline
    $\exists x \in U: A(x)$
      & Gib ein bestimmtes $x \in U$ an und zeige $A(x)$.
      \\ \hline
    $U \subseteq V$
      & Sei $x \in U$. Zeige, dass $x \in V$.
      \\ \hline
    $U = V$
      & Zeige $U \subseteq V$ und $V \subseteq U$.
      \\ \hline
    $\forall n \in \mathbb{N}, n \geq n_0: A(n)$
      & Zeige $A(n)$ und $\forall n \geq n_0: A(n) \implies A(n+1)$
      \\ \hline
  \end{tabular}
  \end{center}
\end{frame}

\end{document}
