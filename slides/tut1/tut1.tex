\documentclass[t]{beamer}
%\documentclass{beamer}
\listfiles

\mode<presentation>
{
  \usetheme[deutsch,titlepage0]{KIT}
% \usetheme[usefoot]{KIT}
% \usetheme{KIT}

%%  \usefonttheme{structurebold}

  \setbeamercovered{transparent}

  %\setbeamertemplate{enumerate items}[circle]
  \setbeamertemplate{enumerate items}[ball]

}
\usepackage{babel}
%\date{10.05.2010}
%\DateText

\newlength{\Ku}
\setlength{\Ku}{1.43375pt}

\usepackage[utf8]{inputenc}
\usepackage[TS1,T1]{fontenc}
\usepackage{array}
\usepackage{multicol}
\usepackage{xspace}
\usepackage{listings}
\usepackage[sort&compress]{natbib}
\usepackage{inconsolata}
\usepackage[scaled]{beramono}

%\usenavigationsymbols
%\usenavigationsymbols[sfHhdb]
%\usenavigationsymbols[sfhHb]

\input{../config}

\title[]{Tutorium Algorithmen I - Laufzeitanalyse}
\subtitle{\authorName}

\author[]{\authorName}

\AuthorTitleSep{}

\institute[]{INSTITUT FÜR THEORETISCHE INFORMATIK, PROF. SANDERS}

\TitleImage[width=\titleimagewd]{title}

\newlength{\tmplen}

\newcommand{\verysmall}{\fontsize{6pt}{8.6pt}\selectfont}

\newcommand{\Rplus}{\protect\raisebox{.1ex}{+}}
\newcommand{\Cpp}{\mbox{C\Rplus\Rplus}\xspace}
\newcommand{\CppEleven}{\mbox{C\Rplus\Rplus11}\xspace}
\newcommand{\CppNext}{\mbox{C\Rplus\Rplus1x}\xspace}

\lstset{basicstyle=\normalsize\ttfamily,
        numbers=left,
        numberstyle=\scriptsize,
        numbersep=-5pt,
        tabsize=4,
        captionpos=b,
        framexleftmargin=1mm,
        xleftmargin=10mm,
        escapeinside={\%*}{*)},
  }
\lstset{literate=%
  {Ö}{{\"O}}1
  {Ä}{{\"A}}1
  {Ü}{{\"U}}1
  {ß}{{\ss}}2
  {ü}{{\"u}}1
  {ä}{{\"a}}1
  {ö}{{\"o}}1
}

\bibliographystyle{plain}

\KITfoot{\parbox[t]{150mm}{
  \vspace{0.1em} \tiny
  \today
  \qquad\qquad\qquad\qquad\qquad\qquad\qquad\qquad \textbf{Tutorium 1}
  \qquad\qquad\qquad\qquad\qquad\qquad\qquad\qquad Algo 1
}}

\begin{document}

\begin{frame}
  \maketitle
\end{frame}

\begin{frame}
  \frametitle{Euer Tutor}
  \begin{itemize}
    \item Name: \authorName
    \item Bachelor Informatik, 4. Semester
    \item E-Mail: \authorEmail
      \begin{itemize}
      \item Auch bei Fragen gerne schreiben!
      \end{itemize}
    \item Diese Folien: \authorHomepage
  \end{itemize}
\end{frame}

\begin{frame}
  \frametitle{Tutorium: Sinn und Zweck}
  \begin{itemize}
    \item \emph{Freiwillige} M"oglichkeit, Vorlesungsstoff zu wiederholen und
          an Beispielen zu "uben („Learning by doing“)
    \item Tutor Ansprechpartner bei Fragen, Schnittstelle zwischen euch und
          "Ubungsleitung/Professor
      \begin{itemize}
      \item Also Fragen stellen, sofort nachdem sie auftreten!
      \end{itemize}
    \item Hinarbeiten auf "Ubungsbl"atter, z.T. gemeinsames Besprechen von L"osungen
    \item Vorrechnen von Aufgaben f"ur Bonuspunkte
          (einmalig bis zu 6 zus"atzliche "Ubungspunkte)
    \item Vorbereitung auf (Zwischen-)Klausur
  \end{itemize}
\end{frame}

\begin{frame}
  \frametitle{Klausurbonus}
  \begin{itemize}
    \item "Ubungspunkte linear interpoliert auf $\leq 4$ Klausurbonuspunkte
          (Klausur hat insgesamt 60P)
      \begin{itemize}
      \item $\geq 20$ "UP $\rightarrow$ 1 BP
      \item $\geq 40$ "UP $\rightarrow$ 2 BP
      \item $\geq 60$ "UP $\rightarrow$ 3 BP
      \item $\geq 80$ "UP $\rightarrow$ 4 BP
      \end{itemize}
    \item "UP setzen sich zusammen aus
      \begin{itemize}
      \item "Ubungsbl"atterpunkte
      \item Mittsemesterklausur (ca. 2 "Ubungsbl"atter wert)
      \item Vorrechnen im Tutorium (einmalig $\leq 6$ "UP)
      \begin{itemize}
        \item Aktuelles oder zuletzt abgegebenes "UB (falls Musterl"osung noch
              nicht ver"offentlicht)
      \end{itemize}
      \item Zusatzaufgaben auf "Ubungsbl"attern (z"ahlen nicht zu 100\%)
      \item Programmierwettbewerb (z"ahlt nicht zu 100\%)
      \end{itemize}
  \end{itemize}
\end{frame}

\begin{frame}
  \frametitle{"Ubungsbl"atter}
  \begin{itemize}

  \item Jede Woche eins
  \item Ausgabe Mittwoch, Abgabe nach 9 Tagen freitags bis 12:45 im Infobau
  \item Partnerarbeit erlaubt und erw"unscht
    \begin{itemize}
    \item nur \emph{eine} Abgabe pro Team (mit zwei Namen!)
    \item Name(n), Matrikelnummer(n) und Tutoriumsnummer nicht vergessen!
    \end{itemize}
  \item Offensichtlich abgeschrieben $\rightarrow$ 0 Punkte auf gesamtes Blatt
  \item Korrekturkriterien
    \begin{itemize}
    \item Punktevergabe meist auf Blatt aufgeschl"usselt
    \item Noch genauere Aufschl"usselung f"ur Tutoren, zum Zwecke der Einheitlichkeit
    \end{itemize}
  \item Bei Fragen/Unklarheiten/Zweifeln bitte nach dem Tutorium ansprechen

  \end{itemize}
\end{frame}

\begin{frame}
  \frametitle{Sonstige Resourcen}
  \begin{itemize}
  \item Veranstaltungshomepage: \url{http://algo2.iti.kit.edu/AlgorithmenI2013.php}
  \item Diskussionsforum f"ur "Ubungsaufgaben:
     \small \url{https://ilias.studium.kit.edu/goto_produktiv_crs_216182.html}
  \end{itemize}
\end{frame}

\begin{frame}
  \frametitle{"Uberblick Tutorium}
  %\begin{itemize}
  %\end{itemize}
\end{frame}

\begin{frame}[fragile]
  \frametitle{Linearsuche}
  \begin{itemize}
  \item Eingabe: Array $a$ der Gr"o"se $n$ und Element $x$
  \item Ausgabe: Index $i \in \{0,\ldots,n-1\}$ mit $a[i] = x$
                 oder "not found", falls das Element nicht im Array existiert
  \end{itemize}
  \begin{lstlisting}
  Function LINEAR-SEARCH(a, x)
      for i := 0 to a.length - 1
          if x = a[i]
              return i
      return "not found"
  \end{lstlisting}
\end{frame}

\begin{frame}[fragile]
  \frametitle{Bin"are Suche}
  \begin{itemize}
  \item Eingabe: Array $a$ der Gr"o"se $2^k$ mit $k \geq 1$ und Element
                 $x \in a$. \\ $a$ sei aufsteigend sortiert und enthalte
                 keine doppelten Elemente
  \item Ausgabe: Index $i \in \{0,\ldots,2^k-1\}$ mit $a[i] = x$
  \end{itemize}
  \begin{lstlisting}
  Function BINARY-SEARCH(a, x)
      l := 0
      r := a.length    // = %*$2^k$*)
      while r - l > 1
          m := %*$\frac{1}{2}\left(l+r\right)$*)
          if x %*$\leq$*) a[m]
              r := m
          else
              l := m
      return l
  \end{lstlisting}
\end{frame}

\end{document}
